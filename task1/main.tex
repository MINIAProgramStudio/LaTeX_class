%! TEX program = lualatex
\documentclass[14pt]{extarticle}
\usepackage{setspace}
\setstretch{1.5}

%Підтримка української
\usepackage[ukrainian]{babel}

%\usepackage{showframe}
\usepackage[margin=2cm]{geometry}

\usepackage{amsmath}
\usepackage{amssymb}

\usepackage{fontspec}
\setmainfont{Times New Roman}[WordSpace=1]



\usepackage{titlesec} % Взято з https://www.overleaf.com/learn/latex/Sections_and_chapters
\usepackage{etoolbox}

\titleformat{\section}{\LARGE\centering\fontspec{Arial}}{}{0pt}{}
\titleformat{\subsection}{\large\centering}{}{0pt}{}

\pretocmd{\section}{\vspace*{0pt}}{}{}
\pretocmd{\subsection}{}{}{}

\usepackage{amsthm}
\newtheoremstyle{lemma}
    {12pt}          % space above
    {12pt}          % space below
    {\fontspec{Arial}}      % body font
    {0pt}           % indent
    {\bfseries}     % head font
    {.}             % head punctuation
    { }             % space after head
    {}              % head spec (empty = default)
\theoremstyle{lemma}
\newtheorem{lemma}{Лемма}

\newtheoremstyle{proofstyle}
    {12pt}          % space above
    {12pt}          % space below
    {\normalfont}       % body font
    {0pt}           % indent
    {\itshape}      % head font
    {.}             % head punctuation
    { }             % space after head
    {}              % head spec (empty = default)
\theoremstyle{proofstyle}
\newtheorem*{myproof}{Доведення}

\begin{document}
\abovedisplayskip=5pt
\belowdisplayskip=5pt
\section{Рівневий набір гармонійних функцій}
\subsection{А. В. Тор}
\vspace*{15pt}

\quad Для $\theta \in [0, \pi/2[$, розглянемо множини
{
\begin{align*}
    \Sigma_{1,\theta} &= \left\{a \in \mathbb{C} \backslash ]-\infty, -1]: \Re \left(\int_{[1,a]}e^{i\theta}\sqrt{p_a(z)}dz\right)=0\right\};\\
    \Sigma_{-1,\theta} &= \left\{a \in \mathbb{C} \backslash [1, +\infty[ : \Re \left(\int_{[-1,a]}e^{i\theta}\sqrt{p_a(z)}dz\right)=0\right\};\\
    \Sigma_\theta &= \left\{a \in \mathbb{C} \backslash [-1, 1] : \Re \left(\int_{[-1,1]}e^{i\theta}\sqrt{p_a(z)}dz\right)=0\right\};
\end{align*}
}
де $p_a(z)$ --- комплексний многочлен, визначений формулою
$$p_a(z)=(z-a)(z^2-1).$$
\begin{lemma}
Нехай $\theta \in [0, \pi/2[$. 
Тоді кожна з множин $\Sigma_{1,\theta}$ та $\Sigma_{-1.\theta}$ утворюється двома гладкими кривими,
які локально ортогональні відповідно при $z = 1$ та $z = -1$ точніше:
\begin{align*}
    \lim_{\substack{|a|\longrightarrow -1 \\ a\in\Sigma_{-1,\theta}}} \arg (a+1) &= \frac{-2\theta+(2k+1)\pi}{4}, k = 0,1,2,3;\\
    \lim_{\substack{|a|\longrightarrow +1 \\ a\in\Sigma_{1,\theta}}} \arg (a-1) &= \frac{-\theta+k\pi}{4}, k = 0,1,2,3.
\end{align*}
Дві криві що визначають $\Sigma_{1,\theta}$ (відповідно $\Sigma_{-1,\theta}$), 
перетинаються лише при $z = 1$ (відповідно $z = -1$).
Більше того,
для $\theta \notin \left\{0, \frac{\pi}{2}\right\}$,
вони розходяться по-різному для \infty\ в одному з напрямків
$$\lim_{\substack{|a|\longrightarrow +\infty \\ a\in\Sigma_{\pm1,\theta}}} \arg a = \frac{-2\theta+2k\pi}{5}, k = 0,1,2,3,4.$$
Для $\theta = 0$, (відповідно $\theta = \frac{\pi}{2}$), 
один промінь $\Sigma_{1,\theta}$ (відповідно $\Sigma_{-1,\theta}$) розходиться до $z = -1$ (відповідно $z= 1 $).
\end{lemma}

\begin{myproof}
Нехай задано непостійну гармонічну функцію $u$,
визначену в деякі області $\mathcal{D}$ of $\mathbb{C}$.
Критичними точками $u$ є саме ті, де
$$\frac{\partial u}{\partial z} = \frac{1}{2}\left(\frac{\partial u}{\partial x} - i\frac{\partial u}{\partial y}\right) = 0.$$
Вони ізольовані. Якщо $v$ є гармонічним спряженням $u$ у $\mathcal{D}$, скажімо,
$f(z) = u(z) + iv(z)$ аналітична у $\mathcal{D}$, тоді за Коші-Ріманом,
$$f'(z) = 0 \Longleftrightarrow u'(z) = 0$$
Встановлений рівень
$$\Sigma_{z0} = {z \in \mathcal{D} : u(z) = u(z_0)}$$
$u$ через точку $z_0 \in \mathcal{D}$ залежить від поведінки $f$ поблизу $z_0$.
Точніше, якщо $z_0$ є критичною точкою $u$, ($u'(z_0)=0$), 
то існує околиця $\mathcal{U}$ околу $z_0$,
голоморфної функції $g(z)$ визначена на $\mathcal{U}$, така, що
$$\forall z \in \mathcal{U}, f(z) = (z-z_0)^m g(z); g(z)\neq 0.$$
Взявши гілку m-го кореня з $g(z)$, $f$ має локальну структуру
$$f(z) = (h(z))^m, \forall z \in \mathcal{U}.$$
Звідси випливає, що $\Sigma_{z0}$ локально утворена $m$ аналітичними дугами які проходять через $z_0$ 
і перетинаються там під рівними кутами $\pi / m$.
Через регулярну точку $z_0 \in \mathcal{D}$, ($u'(z_0) \neq 0$), теорема про неявну функцію стверджує, 
що $\Sigma_{z0}$ є локально єдиною аналітичною дугою.
Зауважте, що множина рівнів гармонічної функції не може закінчуватися у звичайній точці.

Розглянемо багатозначну функцію
$$f_{1, \theta}(a) = \int\nolimits_1^a e^{i\theta}\sqrt{p_a(t)}dt, a \in \mathbb{C}$$
Інтегруючи вздовж відрізка $[1, a]$, можна припустити, що без втрати загальності, що
\begin{multline}
    f_{1, \theta}(a) = ie^{i\theta}(a-1)^2 \int\nolimits_{0}^1\sqrt{t(1-t)}\sqrt{t(a-1)+2}dt = (a-1)^2 g(a);\\
    g(1) \neq 0.
\label{equation_1}
\end{multline}
Очевидно, що:
$$\forall a \in \mathbb{C} \backslash ]-\infty, -1], \{t(a-1)+2; t\in[0,1]\} = [2,a+1] \subset \mathbb{C} \backslash ]-\infty, 0].$$
Отже, при фіксованому виборі аргументу та квадратного кореня всередині інтеграла, 
$f_{1, \theta}$ та $g$ є однозначними аналітичними функціями в $\mathbb{C} \backslash ]-\infty, -1]$.

Припустимо, що для деяких $a \in \mathbb{C} \backslash ]-\infty, -1], a \neq 1$,
$$u(a) = \Re f_{1, \theta}(a) = 0; f'_{1, \theta}(a) = 0.$$
Тоді,
$$(a-1)^3 g'(a) + 2f_{1, \theta}(a) = 0.$$
Беручи справжні деталі, ми отримуємо
\begin{align*}
    0 &= \int\nolimits_0^1\sqrt{t(1-t)}\Im\left(e^{i\theta}(a-1)^2\sqrt{t(a-1)+2}\right)dt;\\
    0 &= \Re\left((a-1)^3 g'(a)\right) = \int\nolimits_0^1 t\sqrt{t(1-t)}\Im\left(\frac{e^{i\theta}(a-1)^3}{2\sqrt{t(a-1)+2}}\right)dt.
\end{align*}

За неперервністю функцій всередині цих інтегралів на відрізку $[0,1]$,
існують $t_1, t_2 \in [0,1]$ такі що
$$\Im\left(e^{i\theta}(a-1)^2\sqrt{t_1(a-1)+2}\right) = \Im \left(\frac{e^{i\theta}(a-1)^3}{2\sqrt{t_2(a-1)+2}}\right) = 0;$$
а потім
$$e^{2i\theta}(a-1)^4(t_1(a-1)+2) > 0, \left(\frac{e^{2i\theta}(a-1)}{t_2(a-1)+2}\right) > 0.$$
яка не може виконуватись, оскільки, якщо $\Im a > 0$, то
\begin{multline*}
    0 < \arg(t_1(a-1)+2) + \arg((t_2(a-1)+2)) \\
    < 2\arg(a+1) < arg \left((a-1)^2\right)<2\pi.
\end{multline*}
Випадок $\Im a < 0$ є аналогічним, тоді як випадок $a \in \mathbb{R}$ можна легко відкинути.
Таким чином, $a=1$ є єдиною критичною точкою $\Re f_{1, \theta}$.
Оскільки $f''_{1, \theta}(1) = 2g(1) \neq 0$,
вводиио локальну поведінку $\Sigma_{1, \theta}$ поблизу $a = 1$.

Припустимо, що для деяких $\theta \in ]0, \frac{\pi}{2}[$ промінь $\Sigma_{\pm 1, \theta}$ 
розходиться до певного моменту в $]-\infty, -1[;$ або приклад,
$$\left(\overline{\Sigma_{1, \theta}} \backslash \Sigma_{1, \theta}\right) \cap \left\{z \in \overline{\mathbb{C}} : \Im z \geq 0\right\} = \{x_0\}$$
Нехай $\epsilon > 0$ так, що $0<\theta - 2\epsilon$.
Для $a \in \mathbb{C}$ задовольняє $\pi - \epsilon < \arg a < \pi$,
$0 < \theta - 2 \epsilon < \theta + 2 \arg a + \arg\int_0^1\sqrt{t(1-t)}\sqrt{t(a-1)+2}dt < \frac{\pi}{2}+\theta-\frac{\epsilon}{2}<\pi$,
що суперечить \ref{equation_1}
\end{myproof}
\end{document}